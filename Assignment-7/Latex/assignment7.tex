% Inbuilt themes in beamer
\documentclass{beamer}
\usepackage{enumitem}
\usepackage{amsmath}
\usepackage{amssymb}
\usepackage{gensymb}
\usepackage{graphicx}
\usepackage{multicol}

\def\inputGnumericTable{}

\usepackage[latin1]{inputenc}                                 
\usepackage{color}                                            
\usepackage{array}                                            
\usepackage{longtable}                                        
\usepackage{calc}                                             
\usepackage{multirow}                                         
\usepackage{hhline}                                           
\usepackage{ifthen}
\usepackage{caption} 
\captionsetup[table]{skip=3pt}  

% Theme choice:
\usetheme{CambridgeUS}

% Title page details: 
\title{Assignment-7} 
\author{Nwjwr Khungur Brahma (AI20BTECH11016)}
\date{\today}

\begin{document}

\begin{frame}
\titlepage
\end{frame}

\begin{frame}{Outline}
\tableofcontents
\end{frame}

\section{Question}
\begin{frame}{Question}
A telephone occurs at random in the interval $(0,T)$. This means that the probability that it will occur in the interval $0\leq t \leq t_0$ equals $\dfrac{t_0}{T}$. Thus the outcomes of this experiment are all points in the interval $(0,T)$. Then what will be the probability of the event that the call will occur in the interval $(t_1,t_2)$?
\end{frame}

\section{Solution}
\begin{frame}{Solution}
Lets $t_1$ and $t_2$ be time such that $(t_1,t_2) \in (0,T)$.\\
Already given that the probability that the telephone rings at the interval $(0,t_0)$ = $\dfrac{t_0}{T}$\\
Lets take the event of call occurs at interval $(0,t_1)$ as $E$, for the interval $(0,t_2)$ as $F$ and for the interval $(t_1,t_2)$ be $Z$.
\end{frame}

\begin{frame}
$\therefore P(E)= \dfrac{t_1}{T}$ and $P(F)=\dfrac{t_2}{T}$\\
From the above we can say that the event $Z = F-E$\\
\begin{align}
\therefore P(Z) &= P(F-E)\\
&=P(F)-P(E)\\
&=\dfrac{t_2}{T}-\dfrac{t_1}{T}\\
&=\dfrac{t_2-t_1}{T}
\end{align}
\end{frame}
\end{document}