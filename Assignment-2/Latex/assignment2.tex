\documentclass[journal,12pt,twocolumn]{IEEEtran}

\usepackage{enumitem}
\usepackage{amsmath}
\usepackage{amssymb}
\usepackage{graphicx}

\begin{document}
\vspace{3cm}
\title{ASSIGNMENT-2}
\author{Nwjwr Khungur Brahma(AI20BTECH11016)}
\maketitle
\textbf{Question:}
Solve the equation for x:\bigskip\\
$\sin^{\text{-}1}x + \sin^{\text{-}1}(1-x)= \cos^{\text{-}1}x, x \neq 0$\bigskip \\
\textbf{Solution:}
Taking the equation:
\begin{align}
\sin^{\text{-}1}x+\sin^{\text{-}1}(1-x) = \cos^{\text{-}1}x\label{eq:1}
\end{align}
Also, we know that
\begin{align}
\cos^{\text{-}1}x+\sin^{\text{-}1}x=\frac{\pi}{2}\label{eq:2}
\end{align}
Using \eqref{eq:1} and \eqref{eq:2} we get,
\begin{align}
\implies \sin^{\text{-}1}x+\sin^{\text{-}1}(1-x) &= \frac{\pi}{2} - \sin^{\text{-}1}x\\
\implies \sin^{\text{-}1}(1-x) &= \frac{\pi}{2} - 2\sin^{\text{-}1}x\\
\implies \sin\left(\frac{\pi}{2}-2\sin^{\text{-}1}x\right)&= (1-x)\\
\implies \cos(2\sin^{\text{-}1}x)&=(1-x)\label{eq:3}
\end{align}
We know that,
\begin{align}
\cos(2y)=1-2\sin^{2}y\label{eq:4}
\end{align}
From equation \eqref{eq:3} and \eqref{eq:4} we get,
\begin{align}
\implies 1-2\sin^{2}(\sin^{\text{-}1}x)&= (1-x)\label{eq:5}
\end{align}
Also,
\begin{align}
\sin(\sin^{\text{-}1}x)=x\label{eq:6}
\end{align}
Now taking \eqref{eq:5} and \eqref{eq:6}
\begin{align}
\implies 1-2x^{2}&=(1-x)\\
\implies 2x^{2}-x&=0\\
\implies x(2x-1)&=0\\
\implies x&=0,\frac{1}{2}
\end{align}
As already mentioned that $x\neq0$.
Therefore $x=\frac{1}{2}$
\begin{figure}[!ht]
\centering
\includegraphics[width=\columnwidth]{fig2.png}
\caption{Graph showing the intersection of $y = \sin^{\text{-}1}x+\sin^{\text{-}1}(1-x)$ and $y = \cos^{\text{-1}}x$}
\label{fig}
\end{figure}
\end{document} 