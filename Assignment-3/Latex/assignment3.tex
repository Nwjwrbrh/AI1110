\documentclass[journal,12pt,twocolumn]{IEEEtran}

\usepackage{enumitem}
\usepackage{amsmath}
\usepackage{amssymb}
\usepackage{gensymb}
\usepackage{graphicx}
\usepackage{multicol}

\def\inputGnumericTable{}

\usepackage[latin1]{inputenc}                                 
\usepackage{color}                                            
\usepackage{array}                                            
\usepackage{longtable}                                        
\usepackage{calc}                                             
\usepackage{multirow}                                         
\usepackage{hhline}                                           
\usepackage{ifthen}
\usepackage{caption} 
\captionsetup[table]{skip=3pt}  

\renewcommand{\thefigure}{\arabic{table}}
\renewcommand{\thetable}{\arabic{table}} 
\begin{document}
\vspace{3cm}
\title{ASSIGNMENT-3}
\author{Nwjwr Khungur Brahma(AI20BTECH11016)}
\maketitle
\textbf{Question:}
The following data on the number of girls (to the nearest ten) per thousand boys in different sections of Indian society is given below.
\begin{table}[!ht]
  \input{Table/Table.tex}
\end{table}
\begin{enumerate}
\item Represent the information above by a bar graph.
\item In the classroom discuss what conclusions can be arrived at from the graph.
\end{enumerate}
\textbf{Solution:}
\begin{enumerate}
\item The bar graph is shown below,
\begin{figure}[!ht]
\centering
\includegraphics[scale=0.68]{fig3.png}
\caption{Bar graph showing Section Vs Number of girls per thousand boys}
\end{figure}
\hspace{3cm}
\item The conclusions is that the number of girls going to school is less in every section compared to that of the boys and graph gives us a pictorial depiction of that.
\end{enumerate}
\end{document}